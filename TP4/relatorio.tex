
\documentclass{article}
\usepackage[brazil]{babel}
\usepackage[utf8]{inputenc}

\title{\textbf{Relatorio TP3}}
\author{\textbf{Grupo 22} \\
  David Burth Kurka (070589) \\
  Felipe Eltermann Braga (070803) \\
  Vítor Augusto Wolf Antonioli (072622)}
\date{\today}
\begin{document}

\maketitle

\section{Descricao do trabalho}\

Neste quarto trabalho prático fizemos significativas mudanças relativas à estrutura dos arquivos e dos dados trabalhados durante a execução do programa. Antes grande parte dos registros usados, como as chaves primárias e parte das chaves secundárias era armazenada em memória RAM. Agora, nesse projeto, todas as estruturas ficam armazenadas em H arquivos, ficando durante a execução alocado em memoria exclusivamente as estruturas usadas para uma determinada funcionalidade.
Além disso, cada obra de arte recebe um descritor, permitindo busca de imagens por similaridade.
Fora essas novas características, as principais funcionalidades dos TPs antigos foram mantidas.

Dessa forma, o programa ja conta com as seguintes funcionalidades: 
- insercao de registro em uma base de dados; 
- consulta por registro pelo nome inteiro (chave primaria); 
- consulta por registro por qualquer palavra do titulo, tipo, autor ou ano (chaves secundarias); 
- busca por obras similares (usando descritores)



\section{Especificacoes do programa}\

Quando começa a execução do programa, é verificado a existência de algum arquivo onde estariam armazenadas as chaves primárias ou secundárias.
Se essa busca retornar que não existem arquivos com essas informações, então as chaves primárias (PKs) e as chaves secundárias (SKs) de titulo, tipo, autor e ano são lidas de uma base de dados (se existir uma) e armazenadas em arquivos tanto para consultas durante a execução, quanto para futuras utilizações do programa.
Além disso, é calculado o númer de registros presentes na base, para auxiliar no decorrer do programa.



Depois dessa etapa, o usuario pode escolher entre as seguintes opções:

\textbf{1) Insercao de registro na base de dados:}

A inserção de registro continua seguindo o princípio dos outros TPs, adicionando o registro lido ao arquivo da base e extraindo as chaves primárias e secundárias para possibilitar futuras consultas envolvendo essa nova obra. A diferença agora está no fato de que tanto a chave primária quanto as secundárias são armazenadas em arquivos. A escolha desses arquivos é feita por uma \textbf{função de hash}, que visa distribuir as chaves em H arquivos.



\textbf{2) Listagem de obras presentes no banco de dados:}

Essa função mantém as mesmas especificações dos outros TPs, com a diferença de que ao invés de se ter um vetor com todas as chaves primárias carregads, agora carregamos as chaves primárias de cada arquivo do hash, imprimimos no arquivo HTML os seus dados, liberamos a memória usada e procuramos outro arquivo (se existir).



\textbf{3) Pesquisa por chave primaria:}

Segue as mesmas especificações do último trabalho prático, mas com as chaves sendo lidas de um arquivo e não de um vetor dinâmico. Quando o usuário digita a chave que quer pesquisar, calculamos o \textbf{hash} correspondente à chave e assim conseguimos prever qual o arquivo em que estará localizada a chave, se ela existir. Então é feita a busca exclusivamente no arquivo e retornado o resultado. 



\textbf{4 - 7) Pesquisa por chave secundaria:}

A pesquisa por chave secundaria é feita individualmente para cada campo do registro (titulo, tipo, autor e ano) e tem o funcionamento análogo aos trabalhos anteriores, utilizando agora a mesma estrutura de arquivos da pesquisa por chave primária.


\textbf{8) Procurar por obras similares (descritores):}

Novidade nesse projeto, cada imagem ganha um descritor associado, que são separados em arquivos distintos, dependendo do numero de ``1''s de sua representação binária (o descritor é um char de 8 bits).
Desse modo, imagens com descritores próximos possuem similaridades. O programa é capaz de identificar N imagens mais similares, deixando o usuário escolher a quantidade de imagens que deseja visualizar.


\section{função de hash}\

Algo que merece destaque nesse trabalho, apesar de não ser visível para o usuário final, é o uso de uma função de hash para espalhamento dos arquivos que armazenam chaves primárias e chaves secundárias (arquivos de titulos, tipos, autores e anos).
Quando uma chave é lida da base ou inserida pelo usuário se faz uma operação descrita na função calculaHash, do arquivo ``fopen.c'', que retorna um número entre 0 e NUM_HASH-1, onde NUM_HASH é uma constante definida em defines.h. Esse número é concatenado a uma string que define o tipo de estrutura a qual o hash está sendo calculado e a uma terminação .dat, fazendo a função retornar o nome do arquivo (no formato <TIPOESTRUTURA><HASH>.dat) no qual serão efetuadas as operações de armazenamento ou consulta.
A operação de hash é bastante simples: somam-se os valores ascii dos caracteres da chave e desse resultado, é calculado o resto da divisão por NUM_HASH, obtendo um valor dentro dos limites desejados.
Apesar de simples, a operação demonstra uma boa capacidade de espalhamento, segundo demonstrado na comparação do tamanho dos arquivos gerados, a partir da base00.dat, com NUM_HASH valendo 4:

-rw-r--r-- 1 kurka kurka  13936 2008-06-19 18:18 li_anos0.dat
-rw-r--r-- 1 kurka kurka  12480 2008-06-19 18:18 li_anos1.dat
-rw-r--r-- 1 kurka kurka  22256 2008-06-19 18:18 li_anos2.dat
-rw-r--r-- 1 kurka kurka  12480 2008-06-19 18:18 li_anos3.dat

-rw-r--r-- 1 kurka kurka  20800 2008-06-19 18:18 li_autores0.dat
-rw-r--r-- 1 kurka kurka  32864 2008-06-19 18:18 li_autores1.dat
-rw-r--r-- 1 kurka kurka  35568 2008-06-19 18:18 li_autores2.dat
-rw-r--r-- 1 kurka kurka  18720 2008-06-19 18:18 li_autores3.dat

-rw-r--r-- 1 kurka kurka  25792 2008-06-19 18:18 li_tipos0.dat
-rw-r--r-- 1 kurka kurka   9152 2008-06-19 18:18 li_tipos1.dat
-rw-r--r-- 1 kurka kurka  30784 2008-06-19 18:18 li_tipos2.dat
-rw-r--r-- 1 kurka kurka  36192 2008-06-19 18:18 li_tipos3.dat

-rw-r--r-- 1 kurka kurka  32240 2008-06-19 18:18 li_titulos0.dat
-rw-r--r-- 1 kurka kurka  48048 2008-06-19 18:18 li_titulos1.dat
-rw-r--r-- 1 kurka kurka  35984 2008-06-19 18:18 li_titulos2.dat
-rw-r--r-- 1 kurka kurka  43264 2008-06-19 18:18 li_titulos3.dat

-rw-r--r-- 1 kurka kurka  13104 2008-06-19 18:18 pk0.dat
-rw-r--r-- 1 kurka kurka  13728 2008-06-19 18:18 pk1.dat
-rw-r--r-- 1 kurka kurka  16640 2008-06-19 18:18 pk2.dat
-rw-r--r-- 1 kurka kurka  17680 2008-06-19 18:18 pk3.dat

-rw-r--r-- 1 kurka kurka    632 2008-06-19 18:18 sk_anos0.dat
-rw-r--r-- 1 kurka kurka    385 2008-06-19 18:18 sk_anos1.dat
-rw-r--r-- 1 kurka kurka    554 2008-06-19 18:18 sk_anos2.dat
-rw-r--r-- 1 kurka kurka    450 2008-06-19 18:18 sk_anos3.dat

-rw-r--r-- 1 kurka kurka   1009 2008-06-19 18:18 sk_autores0.dat
-rw-r--r-- 1 kurka kurka   1075 2008-06-19 18:18 sk_autores1.dat
-rw-r--r-- 1 kurka kurka    872 2008-06-19 18:18 sk_autores2.dat
-rw-r--r-- 1 kurka kurka    748 2008-06-19 18:18 sk_autores3.dat

-rw-r--r-- 1 kurka kurka    449 2008-06-19 18:18 sk_tipos0.dat
-rw-r--r-- 1 kurka kurka    361 2008-06-19 18:18 sk_tipos1.dat
-rw-r--r-- 1 kurka kurka    494 2008-06-19 18:18 sk_tipos2.dat
-rw-r--r-- 1 kurka kurka    405 2008-06-19 18:18 sk_tipos3.dat

-rw-r--r-- 1 kurka kurka   2064 2008-06-19 18:18 sk_titulos0.dat
-rw-r--r-- 1 kurka kurka   1705 2008-06-19 18:18 sk_titulos1.dat
-rw-r--r-- 1 kurka kurka   1997 2008-06-19 18:18 sk_titulos2.dat
-rw-r--r-- 1 kurka kurka   2120 2008-06-19 18:18 sk_titulos3.dat

Note os valores dos arquivos li_titulos*.dat, que possuem registros de tamanhos fixos:
-rw-r--r-- 1 kurka kurka  32240 2008-06-19 18:18 li_titulos0.dat
-rw-r--r-- 1 kurka kurka  48048 2008-06-19 18:18 li_titulos1.dat
-rw-r--r-- 1 kurka kurka  35984 2008-06-19 18:18 li_titulos2.dat
-rw-r--r-- 1 kurka kurka  43264 2008-06-19 18:18 li_titulos3.dat

A diferença entre o tamanho maior arquivo (li_titulos1.dat) e o menor (li_titulos0.dat) é de apenas 11024 bytes e cada arquivo (indo de li_titulos0... até li_titulos3..) possui respectivamente 20.2\%, 30,1\%, 22,6\% e 27,1\% do tamanho total.


\section{Problemas}\

*Neste trabalho não foram implementadas funções de remoção de registros. Desse modo, a opção de remover registros foi removida do projeto, não constando essa opção ao usuário.
*Possivel vazamento de memória: rodando o valgrind ele acusa alguns erros, mas o número de mallocs e frees sao iguais. Não consegui retirar esses avisos.
\end{document}
