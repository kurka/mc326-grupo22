\documentclass{article}
\usepackage[brazil]{babel}
\usepackage[utf8]{inputenc}

\title{\textbf{Relatorio TP3}}
\author{\textbf{Grupo 22} \\
  David Burth Kurka (070589) \\
  Felipe Eltermann Braga (070803) \\
  Vítor Augusto Wolf Antonioli (072622)}
\date{\today}
\begin{document}

\maketitle

\section{Descricao do trabalho}\

Implementamos, neste terceiro trabalho pratico, um sistema de busca por indice secundario e remocao de registros de uma base de dados, alem de manter todas as funcionalidades dos trabalhos praticos anteriores e corrigir alguns problemas (o mais serio: vetor dobravel).

Dessa forma, o programa ja conta com as seguintes funcionalidades: 
- insercao de registro em uma base de dados; 
- consulta por registro pelo nome (chave primaria); 
- consulta por registro por qualquer palavra (chave secundaria) (não funcionando totalmente); 
- remocao de registro.

A busca por chave secundaria se da a partir da comparacao com cada palavra dos seguintes campos do registro: titulo, autor, tipo e ano. Para a remocao, implementamos um sistema de gerenciamento de memoria em disco a partir de uma lista invertida. Mais especificacoes serao dadas adiante.

\section{Especificacoes do programa}\

Antes mesmo do usuario escolher a primeira opcao da interface, o programa faz a execucao da seguinte rotina: 
- aloca memoria para o vetor de registros (chave primaria + NRR); 
- abre (ou cria se não existir) o arquivo base.dat;
- abre (ou cria se não existir) o arquivo com o indice da avail list (-1 se nao existir avail list);
- tenta abrir o arquivo pk.dat: caso este exista carrega as chaves primarias para a memoria em vetor-registros, caso contrario inicia-o a partir dos registros na base (se houver);
- se a leitura de chaves primárias foi feita do arquivo base.dat, verifica se dentro do arquivo existe avail list. Em caso verdadeiro, elimina do vetor de registros os campos existentes;
- constroi arquivos de chaves secundarias, com listas ligadas de palavras chaves;
- calcula o numero de registros trabalhados;


Depois dessa etapa, o usuario pode escolher entre as seguintes opções:

1)Criacao/adicao do banco de dados:

-Alem das funcionalidades dos TPs anteriores, sempre que uma obra é adicionada, é conferido se existem espaços no arquivo base.dat já existente (através da avail list). 
Se houver espaço, o novo registro é inserido no topo da pilha indicado em avail list. Se não houver espaço, o novo registro é inserido no final do arquivo base.dat.
As chaves primarias são atualizadas junto com os registros, recebendo corretamente o número relativo de registro (NRR) de onde o registro está sendo inserido.

2)Listagem de obras presentes no banco de dados:

-Essa função mantém as mesmas especificações dos outros TPs;

3)Busca de obras no banco de dados por chave primária;

-Essa função mantém as mesmas especificações dos outros TPs;

4)Busca de obras no banco de dados por chave secundaria;



5)Remoção de obras.

-Função implementada nesse TP, faz a leitura de uma chave primária dada pelo usuário e, achando-a remove do arquivo base.dat e do vetor de chaves primárias (que será transformado em pk.dat).
Quando remove do arquivo base.dat, forma uma avail list, para reaproveitamento do espaço no momento da inserção.


\section{Problemas}\

/* ... */

\end{document}
