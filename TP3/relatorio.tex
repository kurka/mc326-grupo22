\documentclass{article}
\usepackage[brazil]{babel}
\usepackage[utf8]{inputenc}

\title{\textbf{Relatorio TP3}}
\author{\textbf{Grupo 22} \\
  David Burth Kurka (070589) \\
  Felipe Eltermann Braga (070803) \\
  Vítor Augusto Wolf Antonioli (072622)}
\date{\today}
\begin{document}

\maketitle

\section{Descricao do trabalho}\

Implementamos, neste terceiro trabalho pratico, um sistema de busca por indice secundario e remocao de registros de uma base de dados, alem de manter todas as funcionalidades dos trabalhos praticos anteriores e corrigir alguns problemas (o mais serio: vetor dobravel).

Dessa forma, o programa ja conta com as seguintes funcionalidades: 
- insercao de registro em uma base de dados; 
- consulta por registro pelo nome (chave primaria); 
- consulta por registro por qualquer palavra (chave secundaria); 
- remocao de registro.

A busca por chave secundaria se da a partir da comparacao com cada palavra dos seguintes campos do registro: titulo, autor, tipo e ano. Para a remocao, implementamos um sistema de gerenciamento de memoria em disco a partir de uma lista invertida. Mais especificacoes serao dadas adiante.

\section{Especificacoes do programa}\

Antes mesmo do usuario escolher a primeira opcao da interface, o programa faz a execucao da seguinte rotina: 
- aloca memoria para o vetor de registros (chave primaria + NRR); 
- abre (ou inicia) o arquivo base22.dat e calcula o numero de registros ja existentes na base no inicio do programa; 
- tenta abrir o arquivo pk.dat: caso este exista carrega as chaves primarias para a memoria em vetor-registros, caso contrario inicia-o a partir dos registros na base (se houver).



\section{Problemas}\

/* ... */

\end{document}
